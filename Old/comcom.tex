\ProvidesPackage{comcom}[1.0, 2008/01/01, Adam Walsh]
%Provides common commands, environments and similar
%Written by Adam Walsh
%Version v1.0
%Version 1.0 re-written in order to provide much of the functionality of the physics package
%Not backwards compatable at all
%Large portions of this package were ripped off (sometimes the ideas and sometimes the code) from other packages
%This is the ancient version of the fancycom package
%The package is slightly broken but functionally works
%10/10 would not suggest using anywhere except as reference

\usepackage{amsfonts}
\usepackage{amsthm}
\usepackage{amsmath}
\usepackage{amssymb}
\usepackage[usenames,dvipsnames]{color}
\usepackage{graphicx}
\usepackage{calc}
\usepackage{textcomp}
\usepackage{wasysym}
\usepackage{xparse}

%The command shortcuts

%General utilities
\newcommand{\fnewline}{\color{White}.\color{black}\newline} %Gives me a forced newline
\newcommand{\what}[1]{& \qquad &\begin{sloppy}\textit{\small\mbox{#1}\small}\end{sloppy}} %Little comment for an align block
\newcommand{\dsp}{\displaystyle} %A way to make inline integrals and other limits stack correctly
\newcommand{\hitem}[1]{\item \textbf{#1:}\\} %Heads the item in an itemize/enumerate

%Textual shortcuts/aids
\newcommand{\bb}[1]{\textbf{#1}} %Embolden your words!
\newcommand{\mbb}[1]{\mathbb{#1}} %Embolden your symbols!
\newcommand{\cndall}{\,\,\,\forall} %forall given as condition
\newcommand{\hint}{\textit{Hint.}  } %Measure theory textbook style hint.

%Symbol abbreviations
\renewcommand{\qed}{\nobreak \ifvmode \relax \else \ifdim\lastskip<1.5em \hskip-\lastskip \hskip1.5em plus0em minus0.5em \fi \nobreak\CheckedBox\fi} %A qed symbol
\newcommand{\ceq}{\coloneqq} %Defined to be equal to
\newcommand{\eqc}{\eqqcolon} %Equal to be defined to
\newcommand{\ra}{\rangle} %Right angle bracket
\newcommand{\la}{\langle} %Left angle bracket
\newcommand{\wto}{\rightharpoonup} %Weak convergence, or just an arrow with the bottom half missing
\newcommand{\vphi}{\varphi} %I am sick of writing \varphi
\renewcommand{\to}{\longrightarrow} %A long right arrow, for functions and such
\newcommand{\lmto}{\longmapsto} %A right arrow with a vertical tail, for mapping definitions
\renewcommand{\d}{\,d} %Don't you ever get tired of unformatted integrals?
\renewcommand{\a}{\approx} %curvy equals.. I think
\newcommand{\bsl}{\backslash} %A less cumbersome way of writing backslash
\newcommand{\st}{\,\,\,\big|\,\,\,} %such that
\newcommand{\llim}{\lim\limits} %limit set up so that it always writes the limits underneath
\newcommand{\linf}{\llim_{n\to\infty}} %Limit as n goes to infinity
\newcommand{\lidiv}{\llim_{h\to 0}} %Limit for the definition of a derivative.
\newcommand{\divides}{\,|\,} %a | b, divides symbol
\newcommand{\ndivides}{\,\nmid\,} %a does not divide b, not-divides symbol
\newcommand{\nin}{\notin} %element symbol with slash through it

%More advanced relations, symbols, and stuffs
\newcommand{\isom}{\backsimeq} %Isometry of groups
\newcommand{\nsubeq}{\unlhd} %Normal subgroup possibly equal
\newcommand{\nsub}{\lhd} %Normal subgroup not equal

%Spaces and stuff... symbols with arguments
\newcommand{\gl}[1]{\bb{GL}_#1(\mbb{R})} %General Linear group GL_n(R)
\newcommand{\coll}[2]{\{#1\}_{#2}} %A slightly simpler way of writing collections of sets
\DeclareDocumentCommand\flatfrac{ m m }{\left.#1\middle\slash#2\right.}

%Stuff to construct symbols


%Function constructors
\newcommand{\func}[3]{#1:\,\,\,#2\to #3} %Generic function definition without mapping
\newcommand{\vfunc}[5]{\func{#1}{#2}{#3},\quad#4\longmapsto #5} %Generic function definition with mapping
\newcommand{\vlfunc}[5]{#1:\,\,\,&#2\to #3,\\&#4\longmapsto #5} %No idea
\newcommand{\rnfunc}[2]{\func{#1}{\mbb{R}^#2}{\mbb{R}^#2}} %Function mapping from R^n to R^n
\newcommand{\rnnfunc}[3]{\func{#1}{\mbb{R}^#2}{\mbb{R}^#3}} %Function mapping from R^n to R^m
\newcommand{\cndfunc}[4]{#1:\,\,\,#2\to #3,\quad#4} %Function with name, domain, codomain and condition

%Other constructors
\renewcommand{\mod}[3]{#1 \,\,\,\equiv \,\,\,#2\,\,\,(\text{mod}\,\,\, #3)} %A rather confusing way of writing a \equiv b (mod c)
\newcommand{\alignmod}[3]{#1 \,\,\,&\equiv \,\,\,#2\,\,\,(\text{mod}\,\,\, #3)} %Writing a &\equiv b (mod c) for an align block. Rather specialised if you ask me.


%Large summand operator things
\newcommand{\aint}[3]{\int_{#1}^{#2}{\left( #3 \right) }} %Writes integral with brackets around integrand
\newcommand{\dint}[4]{\int_{#1}^{#2}{\left( #3 \right) \d#4}} %Integrals with the d outside the brackets for once!
\newcommand{\acup}[3]{\bigcup_{#1}^{#2}{\left( #3 \right) }} %Union written with limits
\newcommand{\acap}[3]{\bigcap_{#1}^{#2}{\left( #3 \right) }} %Intersection written with limits
\newcommand{\alim}[3]{\lim_{#1\to #2}\big\{ #3 \big\}} %limit with brackets.. the brackets sometimes get in the way
\newcommand{\brac}[3]{\bigg]_{#3=#1}^{#3=#2}} %Brackets to put after definite integration
\newcommand{\asum}[3]{\sum\limits_{#1}^{#2}{\left( #3 \right) }} %Writes sum with brackets around summand

%Derivatives, differential operators, that sort of stuff...


%Some extra symbols
\ProvideDocumentCommand\varE{}{\mathcal{E}} % Curly 'E'
\ProvideDocumentCommand\ordersymbol{}{\mathcal{O}} % Order symbol --> O(x^2)
\ProvideDocumentCommand\lparen{}{(} % Left parenthesis
\ProvideDocumentCommand\rparen{}{)} % Right parenthesis


% Brackets and braces
\DeclareDocumentCommand\quantity{}{{\ifnum\z@=`}\fi\@quantity}
\DeclareDocumentCommand\@quantity{ t\big t\Big t\bigg t\Bigg g o d() d|| }
{ % Flexible automatic bracketing of an expression in () or [] or {} or ||
	% Handles manual override of sizing
	\IfBooleanTF{#1}{\let\ltag\bigl \let\rtag\bigr}{
		\IfBooleanTF{#2}{\let\ltag\Bigl \let\rtag\Bigr}{
			\IfBooleanTF{#3}{\let\ltag\biggl \let\rtag\biggr}{
				\IfBooleanTF{#4}
				{\let\ltag\Biggl \let\rtag\Biggr}
				{\let\ltag\left \let\rtag\right}
			}
		}
	}
	% Handles actual bracketing
	\IfNoValueTF{#5}{
		\IfNoValueTF{#6}{
			\IfNoValueTF{#7}{
				\IfNoValueTF{#8}
				{()}
				{\ltag\lvert{#8}\rtag\rvert}
			}
			{\ltag(#7\rtag) \IfNoValueTF{#8}{}{|#8|}}
		}
		{\ltag[#6\rtag] \IfNoValueTF{#7}{}{(#7)} \IfNoValueTF{#8}{}{|#8|}}
	}
	{\ltag\lbrace#5\rtag\rbrace  \IfNoValueTF{#6}{}{[#6]} \IfNoValueTF{#7}{}{(#7)} \IfNoValueTF{#8}{}{|#8|}}
	\ifnum\z@=`{\fi}
}
\DeclareDocumentCommand\qty{}{\quantity} % Shorthand for \quantity
\DeclareDocumentCommand\pqty{ l m }{\braces#1{\lparen}{\rparen}{#2}}
\DeclareDocumentCommand\bqty{ l m }{\braces#1{\lbrack}{\rbrack}{#2}}
\DeclareDocumentCommand\Bqty{ l m }{\braces#1{\lbrace}{\rbrace}{#2}}
\DeclareDocumentCommand\vqty{ l m }{\braces#1{\lvert}{\rvert}{#2}}

%Matrix stuff
\DeclareDocumentCommand\pmqty{m}{\begin{pmatrix}#1\end{pmatrix}}
\DeclareDocumentCommand\Pmqty{m}{\left\lgroup\begin{matrix}#1\end{matrix}\right\rgroup}
\DeclareDocumentCommand\bmqty{m}{\begin{bmatrix}#1\end{bmatrix}}
\DeclareDocumentCommand\vmqty{m}{\begin{vmatrix}#1\end{vmatrix}}
\DeclareDocumentCommand\matrixquantity{}{{\ifnum\z@=`}\fi\@matrixquantity}
\DeclareDocumentCommand\@matrixquantity{ s g o d() d|| }
{
	\mathord{
	\IfNoValueTF{#2}
	{
		\IfNoValueTF{#3}
		{
			\IfNoValueTF{#4}
			{
				\IfNoValueTF{#5}
				{()}
				{\vmqty{#5}}
			}
			{
				\IfBooleanTF{#1}
				{\Pmqty{#4}}
				{\pmqty{#4}}
				\IfNoValueTF{#5}{}{|#5|}
			}
		}
		{\bmqty{#3} \IfNoValueTF{#4}{}{(#4)} \IfNoValueTF{#5}{}{|#5|}}
	}
	{\begin{matrix}#2\end{matrix} \IfNoValueTF{#3}{}{[#3]} \IfNoValueTF{#4}{}{(#4)} \IfNoValueTF{#5}{}{|#5|}}
	}
	\ifnum\z@=`{\fi}
}
\DeclareDocumentCommand\mqty{}{\matrixquantity} % Shorthand for \matrixquantity
\DeclareDocumentCommand\matrixdeterminant{m}{\vmqty{#1}} % Matrix determinant
\DeclareDocumentCommand\mdet{}{\matrixdeterminant} % Shorthand for matrix determinant

%Small matrix stuff
\DeclareDocumentCommand\spmqty{m}{\pqty{\begin{smallmatrix}#1\end{smallmatrix}}}
\DeclareDocumentCommand\sPmqty{m}{\left\lgroup\begin{smallmatrix}#1\end{smallmatrix}\right\rgroup}
\DeclareDocumentCommand\sbmqty{m}{\bqty{\begin{smallmatrix}#1\end{smallmatrix}}}
\DeclareDocumentCommand\svmqty{m}{\vqty{\begin{smallmatrix}#1\end{smallmatrix}}}
\DeclareDocumentCommand\smallmatrixquantity{ s g o d() d|| }
{
	\mathord{
	\IfNoValueTF{#2}
	{
		\IfNoValueTF{#3}
		{
			\IfNoValueTF{#4}
			{
				\IfNoValueTF{#5}
				{()}
				{\svmqty{#5}}
			}
			{
				\IfBooleanTF{#1}
				{\sPmqty{#4}}
				{\spmqty{#4}}
				\IfNoValueTF{#5}{}{|#5|}
			}
		}
		{\sbmqty{#3} \IfNoValueTF{#4}{}{(#4)} \IfNoValueTF{#5}{}{|#5|}}
	}
	{\begin{smallmatrix}#2\end{smallmatrix} \IfNoValueTF{#3}{}{[#3]} \IfNoValueTF{#4}{}{(#4)} \IfNoValueTF{#5}{}{|#5|}}
	}
}
\DeclareDocumentCommand\smqty{}{\smallmatrixquantity} % Shorthand for \smallmatrixquantity
\DeclareDocumentCommand\smallmatrixdeterminant{m}{\svmqty{#1}} % Small matrix determinant
\DeclareDocumentCommand\smdet{}{\smallmatrixdeterminant} % Shorthand for small matrix determinant


\DeclareDocumentCommand\argopen{s}{\IfBooleanTF{#1}{\mathopen{}\mathclose\bgroup}{\mathopen{}\mathclose\bgroup\left}} % Special open grouping for argument of a function
\DeclareDocumentCommand\argclose{s}{\IfBooleanTF{#1}{\egroup}{\aftergroup\egroup\right}} % Special close grouping for argument of a function


\DeclareDocumentCommand\braces{}{{\ifnum\z@=`}\fi\@braces}
\DeclareDocumentCommand\@braces{ s t\big t\Big t\bigg t\Bigg m m m }
{ % General braces with automatic and manual sizing
	\IfBooleanTF{#1}
	{\left#6\smash{#8}\right#7\vphantom{#8}}
	{
		\IfBooleanTF{#2}{\bigl#6{#8}\bigr#7}{
			\IfBooleanTF{#3}{\Bigl#6{#8}\Bigr#7}{
				\IfBooleanTF{#4}{\biggl#6{#8}\biggr#7}{
					\IfBooleanTF{#5}{\Biggl#6{#8}\Biggr#7}{\left#6{#8}\right#7}
				}
			}
		}
	}
	\ifnum\z@=`{\fi}
}

\DeclareDocumentCommand\fbraces{ s t\big t\Big t\bigg t\Bigg m m m m }
{ % Function braces with automatic and manual sizing
	#8
	\IfBooleanTF{#1}
	{\argopen#6\smash{#9}\argclose#7\vphantom{#9}}
	{
		\IfBooleanTF{#2}{\argopen*\bigl#6{#9}\argclose*\bigr#7}{
			\IfBooleanTF{#3}{\argopen*\Bigl#6{#9}\argclose*\Bigr#7}{
				\IfBooleanTF{#4}{\argopen*\biggl#6{#9}\argclose*\biggr#7}{
					\IfBooleanTF{#5}
					{\argopen*\Biggl#6{#9}\argclose*\Biggr#7}
					{\argopen#6{#9}\argclose#7}
				}
			}
		}
	}
}



%Flexible bracket operators
\DeclareDocumentCommand\absolutevalue{ l m }{\braces#1{\lvert}{\rvert}{#2}} % Absolute value/complex modulus
\DeclareDocumentCommand\abs{}{\absolutevalue} % Shorthand for \absolutevalue
\DeclareDocumentCommand\norm{ l m }{\braces#1{\lVert}{\rVert}{#2}} % Norm
\DeclareDocumentCommand\order{ l m }{\fbraces#1{\lparen}{\rparen}{\ordersymbol}{#2}} % Order notation -> O(x^2)


%The eval bar
\DeclareDocumentCommand\evaluated{ s g d[| d(| }
{ % Vertical evaluation bar
	\IfNoValueTF{#2}
	{
		\IfNoValueTF{#3}
		{
			\IfNoValueTF{#4}
				{\argopen.\vphantom{\int}\argclose\rvert}
				{\IfBooleanTF{#1}{\vphantom{#4}}{}\left(\IfBooleanTF{#1}{\smash{#4}}{#4}\vphantom{\int}\right\rvert}
		}
		{\IfBooleanTF{#1}{\vphantom{#3}}{}\left[\IfBooleanTF{#1}{\smash{#3}}{#3}\vphantom{\int}\right\rvert \IfNoValueTF{#4}{}{(#4|}}
	}
	{\IfBooleanTF{#1}{\vphantom{#2}}{}\left.\IfBooleanTF{#1}{\smash{#2}}{#2}\vphantom{\int}\right\rvert \IfNoValueTF{#3}{}{[#3|} \IfNoValueTF{#4}{}{(#4|}}
}
\DeclareDocumentCommand\eval{}{\evaluated} % Shorthand for evaluated


% Vector notation
\DeclareDocumentCommand\vectorbold{ s m }{\IfBooleanTF{#1}{\boldsymbol{#2}}{\mathbf{#2}}} % Vector bold [star for Greek and italic Roman]
\DeclareDocumentCommand\vb{}{\vectorbold} % Shorthand for \vectorbold

\DeclareDocumentCommand\vectorarrow{ s m }{\IfBooleanTF{#1}{\vec{\boldsymbol{#2}}}{\vec{\mathbf{#2}}}} % Vector arrow + bold [star for Greek and italic Roman]
\DeclareDocumentCommand\va{}{\vectorarrow} % Shorthand for \vectorarrow

\DeclareDocumentCommand\vectorunit{ s m }{\IfBooleanTF{#1}{\boldsymbol{\hat{#2}}}{\mathbf{\hat{#2}}}} % Unit vector [star for Greek and italic Roman]
\DeclareDocumentCommand\vu{}{\vectorunit} % Shorthand for \vectorunit

\DeclareDocumentCommand\dotproduct{}{\boldsymbol\cdot} % Vector dot product symbol
\DeclareDocumentCommand\vdot{}{\dotproduct} % Shorthand for \dotproduct [note that the command sequence \dp is protected]

\DeclareDocumentCommand\crossproduct{}{\boldsymbol\times} % Vector cross product symbol
\DeclareDocumentCommand\cross{}{\crossproduct} % Shorthand for \crossproduct
\DeclareDocumentCommand\cp{}{\crossproduct} % Shorthand for \crossproduct



% Quick quad text (math-mode text with \quad spacing)
\DeclareDocumentCommand\qqtext{ s m }{\IfBooleanTF{#1}{}{\quad}\text{#2}\quad}
\DeclareDocumentCommand\qq{}{\qqtext}

\DeclareDocumentCommand\qcomma{}{,\quad}
\DeclareDocumentCommand\qc{}{\qcomma}

\DeclareDocumentCommand\qif{s}{\IfBooleanTF{#1}{}{\quad}\text{if}\quad}
\DeclareDocumentCommand\qthen{s}{\IfBooleanTF{#1}{}{\quad}\text{then}\quad}
\DeclareDocumentCommand\qelse{s}{\IfBooleanTF{#1}{}{\quad}\text{else}\quad}
\DeclareDocumentCommand\qotherwise{s}{\IfBooleanTF{#1}{}{\quad}\text{otherwise}\quad}
\DeclareDocumentCommand\qunless{s}{\IfBooleanTF{#1}{}{\quad}\text{unless}\quad}
\DeclareDocumentCommand\qgiven{s}{\IfBooleanTF{#1}{}{\quad}\text{given}\quad}
\DeclareDocumentCommand\qusing{s}{\IfBooleanTF{#1}{}{\quad}\text{using}\quad}
\DeclareDocumentCommand\qassume{s}{\IfBooleanTF{#1}{}{\quad}\text{assume}\quad}
\DeclareDocumentCommand\qsince{s}{\IfBooleanTF{#1}{}{\quad}\text{since}\quad}
\DeclareDocumentCommand\qlet{s}{\IfBooleanTF{#1}{}{\quad}\text{let}\quad}
\DeclareDocumentCommand\qfor{s}{\IfBooleanTF{#1}{}{\quad}\text{for}\quad}
\DeclareDocumentCommand\qall{s}{\IfBooleanTF{#1}{}{\quad}\text{all}\quad}
\DeclareDocumentCommand\qeven{s}{\IfBooleanTF{#1}{}{\quad}\text{even}\quad}
\DeclareDocumentCommand\qodd{s}{\IfBooleanTF{#1}{}{\quad}\text{odd}\quad}
\DeclareDocumentCommand\qinteger{s}{\IfBooleanTF{#1}{}{\quad}\text{integer}\quad}
\DeclareDocumentCommand\qand{s}{\IfBooleanTF{#1}{}{\quad}\text{and}\quad}
\DeclareDocumentCommand\qor{s}{\IfBooleanTF{#1}{}{\quad}\text{or}\quad}
\DeclareDocumentCommand\qas{s}{\IfBooleanTF{#1}{}{\quad}\text{as}\quad}
\DeclareDocumentCommand\qin{s}{\IfBooleanTF{#1}{}{\quad}\text{in}\quad}
\DeclareDocumentCommand\qcc{s}{\IfBooleanTF{#1}{}{\quad}\text{c.c.}\quad}


%Derivatives! Expanded and revised to match functionality of ``Physics''.
% Derivatives
\DeclareDocumentCommand\differential{ o g d() }{ % Differential 'd'
	% o: optional n for nth differential
	% g: optional argument for readability and to control spacing
	% d: long-form as in d(cos x)
	\IfNoValueTF{#2}{
		\IfNoValueTF{#3}
			{\diffd\IfNoValueTF{#1}{}{^{#1}}}
			{\mathinner{\diffd\IfNoValueTF{#1}{}{^{#1}}\argopen(#3\argclose)}}
		}
		{\mathinner{\diffd\IfNoValueTF{#1}{}{^{#1}}#2} \IfNoValueTF{#3}{}{(#3)}}
	}
\DeclareDocumentCommand\dd{}{\differential} % Shorthand for \differential

\DeclareDocumentCommand\derivative{ s o m g d() }
{ % Total derivative
	% s: star for \flatfrac flat derivative
	% o: optional n for nth derivative
	% m: mandatory (x in df/dx)
	% g: optional (f in df/dx)
	% d: long-form d/dx(...)
	\IfBooleanTF{#1}
	{\let\fractype\flatfrac}
	{\let\fractype\frac}
	\IfNoValueTF{#4}
	{
		\IfNoValueTF{#5}
		{\fractype{\diffd \IfNoValueTF{#2}{}{^{#2}}}{\diffd #3\IfNoValueTF{#2}{}{^{#2}}}}
		{\fractype{\diffd \IfNoValueTF{#2}{}{^{#2}}}{\diffd #3\IfNoValueTF{#2}{}{^{#2}}} \argopen(#5\argclose)}
	}
	{\fractype{\diffd \IfNoValueTF{#2}{}{^{#2}} #3}{\diffd #4\IfNoValueTF{#2}{}{^{#2}}}}
}
\DeclareDocumentCommand\dv{}{\derivative} % Shorthand for \derivative

\DeclareDocumentCommand\partialderivative{ s o m g g d() }
{ % Partial derivative
	% s: star for \flatfrac flat derivative
	% o: optional n for nth derivative
	% m: mandatory (x in df/dx)
	% g: optional (f in df/dx)
	% g: optional (y in d^2f/dxdy)
	% d: long-form d/dx(...)
	\IfBooleanTF{#1}
	{\let\fractype\flatfrac}
	{\let\fractype\frac}
	\IfNoValueTF{#4}
	{
		\IfNoValueTF{#6}
		{\fractype{\partial \IfNoValueTF{#2}{}{^{#2}}}{\partial #3\IfNoValueTF{#2}{}{^{#2}}}}
		{\fractype{\partial \IfNoValueTF{#2}{}{^{#2}}}{\partial #3\IfNoValueTF{#2}{}{^{#2}}} \argopen(#6\argclose)}
	}
	{
		\IfNoValueTF{#5}
		{\fractype{\partial \IfNoValueTF{#2}{}{^{#2}} #3}{\partial #4\IfNoValueTF{#2}{}{^{#2}}}}
		{\fractype{\partial^2 #3}{\partial #4 \partial #5}}
	}
}
\DeclareDocumentCommand\pderivative{}{\partialderivative} % Shorthand for \partialderivative
\DeclareDocumentCommand\pdv{}{\partialderivative} % Shorthand for \partialderivative

\DeclareDocumentCommand\variation{ o g d() }{ % Functional variation
	% o: optional n for nth differential
	% g: optional argument for readability and to control spacing
	% d: long-form as in d(F(g))
	\IfNoValueTF{#2}{
		\IfNoValueTF{#3}
			{\delta \IfNoValueTF{#1}{}{^{#1}}}
			{\mathinner{\delta \IfNoValueTF{#1}{}{^{#1}}\argopen(#3\argclose)}}
		}
		{\mathinner{\delta \IfNoValueTF{#1}{}{^{#1}}#2} \IfNoValueTF{#3}{}{(#3)}}
	}
\DeclareDocumentCommand\var{}{\variation} % Shorthand for \variation

\DeclareDocumentCommand\functionalderivative{ s o m g d() }
{ % Functional derivative
	% s: star for \flatfrac flat derivative
	% o: optional n for nth derivative
	% m: mandatory (g in dF/dg)
	% g: optional (F in dF/dg)
	% d: long-form d/dx(...)
	\IfBooleanTF{#1}
	{\let\fractype\flatfrac}
	{\let\fractype\frac}
	\IfNoValueTF{#4}
	{
		\IfNoValueTF{#5}
		{\fractype{\variation \IfNoValueTF{#2}{}{^{#2}}}{\variation #3\IfNoValueTF{#2}{}{^{#2}}}}
		{\fractype{\variation \IfNoValueTF{#2}{}{^{#2}}}{\variation #3\IfNoValueTF{#2}{}{^{#2}}} \argopen(#5\argclose)}
	}
	{\fractype{\variation \IfNoValueTF{#2}{}{^{#2}} #3}{\variation #4\IfNoValueTF{#2}{}{^{#2}}}}
}
\DeclareDocumentCommand\fderivative{}{\functionalderivative} % Shorthand for \functionalderivative
\DeclareDocumentCommand\fdv{}{\functionalderivative} % Shorthand for \functionalderivative







%OPERATORS

%For automatic flexible braces
\DeclareDocumentCommand\opbraces{ m g o d() }
{
	\IfNoValueTF{#2}
	{
		\IfNoValueTF{#3}
			{
				\IfNoValueTF{#4}
				{#1}
				{\fbraces{\lparen}{\rparen}{#1}{#4}}
			}
			{
				\fbraces{\lbrack}{\rbrack}{#1}{#3}
				\IfNoValueTF{#4}{}{(#4)}
			}
	}
	{
		\fbraces{\lbrace}{\rbrace}{#1}{#2}
		\IfNoValueTF{#3}{}{[#3]}
		\IfNoValueTF{#4}{}{(#4)}
	}
}
\DeclareDocumentCommand\trigbraces{ m o d() }
{
	\IfNoValueTF{#3}
	{#1 \IfNoValueTF{#2}{}{[#2]}}
	{#1 \IfNoValueTF{#2}{}{^{#2}} \argopen(#3\argclose)}
}

%Trig and other operators requiring flexi brackets
	\let\sine\sin \DeclareDocumentCommand\sin{}{\trigbraces{\sine}}
	\let\cosine\cos \DeclareDocumentCommand\cos{}{\trigbraces{\cosine}}
	\let\tangent\tan \DeclareDocumentCommand\tan{}{\trigbraces{\tangent}}
	\let\cosecant\csc \DeclareDocumentCommand\csc{}{\trigbraces{\cosecant}}
	\let\secant\sec \DeclareDocumentCommand\sec{}{\trigbraces{\secant}}
	\let\cotangent\cot \DeclareDocumentCommand\cot{}{\trigbraces{\cotangent}}
	
	\let\arcsine\arcsin \DeclareDocumentCommand\arcsin{}{\trigbraces{\arcsine}}
	\let\arccosine\arccos \DeclareDocumentCommand\arccos{}{\trigbraces{\arccosine}}
	\let\arctangent\arctan \DeclareDocumentCommand\arctan{}{\trigbraces{\arctangent}}
	\DeclareMathOperator{\arccosecant}{arccsc}
	\DeclareDocumentCommand\arccsc{}{\trigbraces{\arccosecant}}
	\DeclareMathOperator{\arcsecant}{arcsec}
	\DeclareDocumentCommand\arcsec{}{\trigbraces{\arcsecant}}
	\DeclareMathOperator{\arccotangent}{arccot}
	\DeclareDocumentCommand\arccot{}{\trigbraces{\arccotangent}}

	\DeclareMathOperator{\asine}{asin}
	\DeclareDocumentCommand\asin{}{\trigbraces{\asine}}
	\DeclareMathOperator{\acosine}{acos}
	\DeclareDocumentCommand\acos{}{\trigbraces{\acosine}}
	\DeclareMathOperator{\atangent}{atan}
	\DeclareDocumentCommand\atan{}{\trigbraces{\atangent}}
	\DeclareMathOperator{\acosecant}{acsc}
	\DeclareDocumentCommand\acsc{}{\trigbraces{\acosecant}}
	\DeclareMathOperator{\asecant}{asec}
	\DeclareDocumentCommand\asec{}{\trigbraces{\asecant}}
	\DeclareMathOperator{\acotangent}{acot}
	\DeclareDocumentCommand\acot{}{\trigbraces{\acotangent}}
	
	\let\hypsine\sinh \DeclareDocumentCommand\sinh{}{\trigbraces{\hypsine}}
	\let\hypcosine\cosh \DeclareDocumentCommand\cosh{}{\trigbraces{\hypcosine}}
	\let\hyptangent\tanh \DeclareDocumentCommand\tanh{}{\trigbraces{\hyptangent}}
	\DeclareMathOperator{\hypcosecant}{csch}
	\DeclareDocumentCommand\csch{}{\trigbraces{\hypcosecant}}
	\DeclareMathOperator{\hypsecant}{sech}
	\DeclareDocumentCommand\sech{}{\trigbraces{\hypsecant}}
	\let\hypcotangent\coth \DeclareDocumentCommand\coth{}{\trigbraces{\hypcotangent}}
	
	\let\exponential\exp \DeclareDocumentCommand\exp{}{\opbraces{\exponential}}
	\let\logarithm\log \DeclareDocumentCommand\log{}{\trigbraces{\logarithm}}
	\let\naturallogarithm\ln \DeclareDocumentCommand\ln{}{\trigbraces{\naturallogarithm}}
	\let\determinant\det \DeclareDocumentCommand\det{}{\opbraces{\determinant}}
	\let\Probability\Pr \DeclareDocumentCommand\Pr{}{\opbraces{\Probability}}
	\DeclareDocumentCommand\tr{}{\opbraces{\trace}}
	\DeclareDocumentCommand\Tr{}{\opbraces{\Trace}}
	\DeclareDocumentCommand\Res{}{\opbraces{\Residue}}


%Mathematical Operators
\DeclareMathOperator*{\lqe}{\le}
\DeclareMathOperator{\rif}{if}
\DeclareMathOperator{\hand}{and}
\DeclareMathOperator{\adj}{adj}
\DeclareMathOperator{\im}{im}
\DeclareMathOperator{\IIm}{Im}
\DeclareMathOperator{\codom}{codom}
\DeclareMathOperator{\dom}{dom}
\DeclareMathOperator{\Aut}{Aut}
\DeclareMathOperator{\inn}{Inn}
\DeclareMathOperator{\jac}{Jac}
\DeclareMathOperator{\image}{Image}
\DeclareMathOperator{\supp}{supp}
\DeclareMathOperator{\lcm}{lcm}
\DeclareMathOperator{\Div}{div}
\DeclareMathOperator{\Mod}{mod}
\DeclareMathOperator{\conv}{conv}



%Matrix macros (Stolen from physics)

% Matrix macros
\DeclareDocumentCommand\identitymatrix{m}
{
	{
		\newtoks\matrixtoks
		\global\matrixtoks = {}
		\newcount\rowcount
		\newcount\colcount
		\loop
		\colcount = 0
		\advance \rowcount by 1
		{
			\loop
			\advance \colcount by 1
			\edef\addtoks
			{
				\ifnum \colcount = 1 \else & \fi
				\ifnum \colcount = \rowcount 1 \else 0 \fi
			}
			\global\matrixtoks = \expandafter{\the\expandafter\matrixtoks\addtoks}
			\ifnum \colcount < #1
			\repeat
		}
		\ifnum \rowcount < #1
			\global\matrixtoks = \expandafter{\the\matrixtoks \\ }
			\repeat
	}
	\the\matrixtoks
}
\DeclareDocumentCommand\imat{}{\identitymatrix}

\DeclareDocumentCommand\xmatrix{ s m m m }
{
	{
		\newtoks\matrixtoks
		\global\matrixtoks = {}
		\newcount\rowcount
		\newcount\colcount
		\loop
		\colcount = 0
		\advance \rowcount by 1
		{
			\loop
			\advance \colcount by 1
			\edef\addtoks{\ifnum \colcount = 1 \else & \fi #2 \IfBooleanTF{#1}{_{\ifnum #3 > 1 \the\rowcount \fi \ifnum #4 > 1 \the\colcount \fi}}{}}
			\global\matrixtoks = \expandafter{\the\expandafter\matrixtoks\addtoks}
			\ifnum \colcount < #4
			\repeat
		}
		\ifnum \rowcount < #3
			\global\matrixtoks = \expandafter{\the\matrixtoks \\ }
			\repeat
	}
	\the\matrixtoks
}
\DeclareDocumentCommand\xmat{}{\xmatrix}

\DeclareDocumentCommand\zeromatrix{ m g }{\IfNoValueTF{#2}{\xmatrix{0}{#1}{#1}}{\xmatrix{0}{#1}{#2}}}
\DeclareDocumentCommand\zmat{}{\zeromatrix}

\DeclareDocumentCommand\paulixmatrix{}{0&1\\1&0}
\DeclareDocumentCommand\pauliymatrix{}{0&-i\\i&0}
\DeclareDocumentCommand\paulizmatrix{}{1&0\\0&-1}
\DeclareDocumentCommand\paulimatrix{m}
{
	\let\argin=#1
	\ifx\argin 0 \identitymatrix{2} \else
	\ifx\argin 1 \paulixmatrix \else
	\ifx\argin 2 \pauliymatrix \else
	\ifx\argin 3 \paulizmatrix \else
	\ifx\argin x \paulixmatrix \else
	\ifx\argin y \pauliymatrix \else
	\ifx\argin z \paulizmatrix \fi\fi\fi\fi\fi\fi\fi
}
\DeclareDocumentCommand\pmat{}{\paulimatrix}

\DeclareDocumentCommand\diagonalmatrix{O{} >{\SplitList{,}}m }{\@dmat{#1}#2}
\DeclareDocumentCommand\@dmat{mmggggggg}
{
	\newtoks\matrixtoks
	\global\matrixtoks = {}
	\IfNoValueTF{#3}
	{#2}
	{
		\IfNoValueTF{#4}
		{\global\matrixtoks = \expandafter{\mqty{#2}&#1\\#1&\mqty{#3}}}
		{
			\IfNoValueTF{#5}
			{\global\matrixtoks = \expandafter{\mqty{#2}&#1&#1\\#1&\mqty{#3}&#1\\#1&#1&\mqty{#4}}}
			{
				\IfNoValueTF{#6}
				{\global\matrixtoks = \expandafter{\mqty{#2}&#1&#1&#1\\#1&\mqty{#3}&#1&#1\\#1&#1&\mqty{#4}&#1\\#1&#1&#1&\mqty{#5}}}
				{
					\IfNoValueTF{#7}
					{\global\matrixtoks = \expandafter{\mqty{#2}&#1&#1&#1&#1\\#1&\mqty{#3}&#1&#1&#1\\#1&#1&\mqty{#4}&#1&#1\\#1&#1&#1&\mqty{#5}&#1\\#1&#1&#1&#1&\mqty{#6}}}
					{
						\IfNoValueTF{#8}
						{\global\matrixtoks = \expandafter{\mqty{#2}&#1&#1&#1&#1&#1\\#1&\mqty{#3}&#1&#1&#1&#1\\#1&#1&\mqty{#4}&#1&#1&#1\\#1&#1&#1&\mqty{#5}&#1&#1\\#1&#1&#1&#1&\mqty{#6}&#1\\#1&#1&#1&#1&#1&\mqty{#7}}}
						{
							\IfNoValueTF{#9}
							{\global\matrixtoks = \expandafter{\mqty{#2}&#1&#1&#1&#1&#1&#1\\#1&\mqty{#3}&#1&#1&#1&#1&#1\\#1&#1&\mqty{#4}&#1&#1&#1&#1\\#1&#1&#1&\mqty{#5}&#1&#1&#1\\#1&#1&#1&#1&\mqty{#6}&#1&#1\\#1&#1&#1&#1&#1&\mqty{#7}&#1\\#1&#1&#1&#1&#1&#1&\mqty{#8}}}
							{\global\matrixtoks = \expandafter{\mqty{#2}&#1&#1&#1&#1&#1&#1&#1\\#1&\mqty{#3}&#1&#1&#1&#1&#1&#1\\#1&#1&\mqty{#4}&#1&#1&#1&#1&#1\\#1&#1&#1&\mqty{#5}&#1&#1&#1&#1\\#1&#1&#1&#1&\mqty{#6}&#1&#1&#1\\#1&#1&#1&#1&#1&\mqty{#7}&#1&#1\\#1&#1&#1&#1&#1&#1&\mqty{#8}&#1\\#1&#1&#1&#1&#1&#1&#1&\mqty{#9}}}
						}
					}
				}
			}
		}
	}
	\the\matrixtoks
}
\DeclareDocumentCommand\dmat{}{\diagonalmatrix}

\DeclareDocumentCommand\antidiagonalmatrix{O{} >{\SplitList{,}}m }{\@admat{#1}#2}
\DeclareDocumentCommand\@admat{mmggggggg}
{
	\newtoks\matrixtoks
	\global\matrixtoks = {}
	\IfNoValueTF{#3}
	{#2}
	{
		\IfNoValueTF{#4}
		{\global\matrixtoks = \expandafter{#1&\mqty{#2}\\\mqty{#3}&#1}}
		{
			\IfNoValueTF{#5}
			{\global\matrixtoks = \expandafter{#1&#1&\mqty{#2}\\#1&\mqty{#3}&#1\\\mqty{#4}&#1&#1}}
			{
				\IfNoValueTF{#6}
				{\global\matrixtoks = \expandafter{#1&#1&#1&\mqty{#2}\\#1&#1&\mqty{#3}&#1\\#1&\mqty{#4}&#1&#1\\\mqty{#5}&#1&#1&#1}}
				{
					\IfNoValueTF{#7}
					{\global\matrixtoks = \expandafter{#1&#1&#1&#1&\mqty{#2}\\#1&#1&#1&\mqty{#3}&#1\\#1&#1&\mqty{#4}&#1&#1\\#1&\mqty{#5}&#1&#1&#1\\\mqty{#6}&#1&#1&#1&#1}}
					{
						\IfNoValueTF{#8}
						{\global\matrixtoks = \expandafter{#1&#1&#1&#1&#1&\mqty{#2}\\#1&#1&#1&#1&\mqty{#3}&#1\\#1&#1&#1&\mqty{#4}&#1&#1\\#1&#1&\mqty{#5}&#1&#1&#1\\#1&\mqty{#6}&#1&#1&#1&#1\\\mqty{#7}&#1&#1&#1&#1&#1}}
						{
							\IfNoValueTF{#9}
							{\global\matrixtoks = \expandafter{#1&#1&#1&#1&#1&#1&\mqty{#2}\\#1&#1&#1&#1&#1&\mqty{#3}&#1\\#1&#1&#1&#1&\mqty{#4}&#1&#1\\#1&#1&#1&\mqty{#5}&#1&#1&#1\\#1&#1&\mqty{#6}&#1&#1&#1&#1\\#1&\mqty{#7}&#1&#1&#1&#1&#1\\\mqty{#8}&#1&#1&#1&#1&#1&#1}}
							{\global\matrixtoks = \expandafter{#1&#1&#1&#1&#1&#1&#1&\mqty{#2}\\#1&#1&#1&#1&#1&#1&\mqty{#3}&#1\\#1&#1&#1&#1&#1&\mqty{#4}&#1&#1\\#1&#1&#1&#1&\mqty{#5}&#1&#1&#1\\#1&#1&#1&\mqty{#6}&#1&#1&#1&#1\\#1&#1&\mqty{#7}&#1&#1&#1&#1&#1\\#1&\mqty{#8}&#1&#1&#1&#1&#1&#1\\\mqty{#9}&#1&#1&#1&#1&#1&#1&#1}}
						}
					}
				}
			}
		}
	}
	\the\matrixtoks
}
\DeclareDocumentCommand\admat{}{\antidiagonalmatrix}



%For theorems and proofs

\newtheorem{Theorem}{Theorem}
\newtheorem{theorem}{Theorem}
\newtheorem{defintion}{Definition}
\newtheorem{definition}{Definition}
\newtheorem{lemma}{Lemma}
\newtheorem{Lemma}{Lemma}
\newtheorem{Note}{Note}
\newtheorem{Example}{Example}


\renewenvironment{proof}[1][Proof]{\begin{trivlist}
\item[\hskip \labelsep {\bfseries #1}]}
{\qed\end{trivlist}}
